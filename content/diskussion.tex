\section{Diskussion}
\label{sec:Diskussion}
Bei allen Auswertungsteilen sind die Ungenauigkeiten zu beachten, welche sich direkt aus dem Versuchsaufbau ergeben und somit zu systematischen
Fehlern in den Ergebnissen führen. Zum Einen ist zu beachten, dass es nicht möglich war mit der Heizapparatur die Temperatur des 
Röhrengehäuses konstant zu halten. Bei jeder Messung traten daher Schwankungen der Temperatur auf. Dies lag zudem auch an dem Heizfaden, welcher
die Apparatur zusätzlich beheizte. Des Weiteren treten Fehler durch das grafische Auswerten des Versuches auf: das Zeichengerät hat eine eingeschränkte 
Genauigkeit, was zum Beispiel an dem wellenartigen Charakter aller Graphen zu sehen ist. Des Weiteren ist die Dicke des verwendeten Stiftes
zu Beachten, da hier schon fast eine Dicke von 1mm (also ein Kästchen auf dem Millimeterpapier) vorliegt, was die Genauigkeit daher stark 
beeinträchtigt. Hinzu kommt außerdem die Genauigkeit mit der das menschliche Auge Werte aus solchen Grafiken ablesen kann.
Das zu Anfang bestimmte Verhältnis $\frac{a}{w}$ zwischen der mittleren freien Weglänge $\overline{w}$ und dem Abstand $a$ zwischen Kathode und Beschleunigungselektrode
liegt im besten Fall zur Erstellung einer Franck-Hertz-Kurve zwischen $1000$ und $4000$. Dies gelang allerdings nur bei der zweiten Messung (bei $143$°C) der Energieverteilung.
Die Messungen der Energieverteilung verliefen erwartungsgemäß. Der Peak bei der ersten Messung ist gut ausgeprägt, was eine 
gute Abschätzung des Kontaktpotentials möglich macht. Das Verhältnis $\frac{a}{w}$ ist hier um eine Faktor $1000$ geringer als es zur Bestimmung
einer Franck-Hertz-Kurve optimal wäre; dadurch regen weniger Elektronen die Quecksilberatome an, wodurch das Ergebnis weniger verfälscht wird.
Anders ist es bei der gleichen Messung bei $143$°C der Fall: hier liegt das erwähnte Verhältnis sehr gut im angesprochenen Bereich, weshalb 
viele Elektronen die Quecksilberatome anregen und zu dem starken Abfall der Anzahl Elektronen über einer Energie von $4$eV führen. 
Durch die Aufnahme von fünf verschiedenen Franck-Hertz-Kurven konnte die für die Auswertung ausgewählt werden, welche die deutlichsten 
Extrema aufweist, was die grafische Auswertung hier etwas vereinfachte. Hier liegt das Verhältnis über dem optimalen Bereich, was wohl 
zu einer noch höheren Stoßwahrscheinlichkeit im Rohr führt. Hierdurch kommt es aber auch zu mehr elastischen Stößen, welche zu einer deutlichen 
Abflachung der Extrema führen. Dies wird besonders bei höher angelegten Spannungen in der Grafik deutlich. Der bestimmte Wert
\\ \\
    \centerline{$E_{\symup{Anr}} = (5.24 \pm 0.11) \symup{eV}$} 
\\ \\
weicht nur um $6.49\%$ vom Literaturwert     
\\ \\
\centerline{$E_{\symup{Lit_{Anr}}} = 4.9 \symup{eV}$\cite{anr}} 
\\ \\
ab, was auf Grund der Fehlerquellen vertretbar ist.
Die beiden Kontaktpotentiale $K_{\symup{1}} = 1.76$V und $K_{\symup{2}} = 0.46$V, weichen um $282.61 \%$ voneinander ab.
Die Ionisierungsenergie $E_{\symup{ion, 1}} = 8.24 \symup{eV}$ weicht um $26.67 \%$ und die Ionisierungsenergie 
$E_{\symup{ion, 2}} = (9.54 \pm 0.11) \symup{eV}$ um $10.45 \%$ vom Literaturwert
\\ \\
    \centerline{$E_{\symup{ion}} = 10.438 \symup{eV}$  \cite{ion}}
\\ \\
ab. Dies lässt vermuten, dass die Messung des zweiten Kontaktpotentials besser ist. Insgesamt ist die starke Abweichung der beiden 
Kontaktpotentiale voneinander allerdings nur durch einen zufälligen Fehler zu erklären, da beide Versuchsteile gut verliefen. 
Die relativ geringe Abweichung der Ionisierungsenergie vom Literaturwert lässt auf eine gute Messung schließen, wobei hier die 
Problematik der grafischen Auswertung bei der Bestimmung der Asymptote besonders zum Tragen kommt.  


