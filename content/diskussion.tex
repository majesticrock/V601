\section{Diskussion}
\label{sec:Diskussion}
Bei allen Auswertungsteilen sind die Ungenauigkeiten zu beachten, welche sich direkt aus dem Versuchsaufbau ergeben und somit zu systematischen
Fehlern in den Ergebnissen führen. Zum Einen ist zu beachten, dass es nicht möglich war mit der Heizapparatur die Temperatur des 
Röhrengehäuses konstant zu halten. Bei jeder Messung traten daher Schwankungen der Temperatur auf. Dies lag zudem auch an dem Heizfaden, welcher
die Apparatur zusätzlich beheizte. Des Weiteren treten Fehler durch das grafische Auswerten des Versuches auf: das Zeichengerät hat eine eingeschränkte 
Genauigkeit, was zum Beispiel an dem wellenartigen Charakter aller Graphen zu sehen ist. Des Weiteren ist die Dicke des verwendeten Stiftes
zu Beachten, da hier schon fast eine Dicke von 1mm (also ein Kästchen auf dem Millimeterpapier) vorliegt, was die Genauigkeit daher stark 
beeinträchtigt. Hinzu kommt außerdem die Genauigkeit mit der das menschliche Auge Werte aus solchen Grafiken ablesen kann.
Das zu Anfang bestimmte Verhältnis $\frac{a}{w}$ zwischen der mittleren freien Weglänge $\overline{w}$ und dem Abstand $a$ zwischen Kathode und Beschleunigungselektrode
liegt im besten Fall zur Erstellung einer Franck-Hertz-Kurve zwischen $1000$ und $4000$. Dies gelang allerdings nur bei der zweiten Messung (bei $143$°C) der Energieverteilung.
Die Messungen der Energieverteilung verliefen erwartungsgemäß. Der Peak bei der ersten Messung ist gut ausgeprägt, was eine 
gute Abschätzung des Kontaktpotentials möglich macht. Das Verhältnis $\frac{a}{w}$ ist hier um eine Faktor $1000$ geringer als es zur Bestimmung
einer Franck-Hertz-Kurve optimal wäre; dadurch regen weniger Elektronen die Quecksilberatome an, wodurch das Ergebnis weniger verfälscht wird.
Anders ist es bei der gleichen Messung bei $143$°C der Fall: hier liegt das erwähnte Verhältnis sehr gut im angesprochenen Bereich, weshalb 
viele Elektronen die Quecksilberatome anregen und zu dem starken Abfall der Anzahl Elektronen über einer Energie von $4$eV führen. 
Durch die Aufnahme von fünf verschiedenen Franck-Hertz-Kurven konnte die für die Auswertung ausgewählt werden, welche die deutlichsten 
Extrema aufweist, was die grafische Auswertung hier etwas vereinfachte. Hier liegt das Verhältnis über dem optimalen Bereich, was wohl 
zu einer noch höheren Stoßwahrscheinlichkeit im Rohr führt. Hierdurch kommt es aber auch zu mehr elastischen Stößen, welche zu einer deutlichen 
Abflachung der Extrema führen. Dies wird besonders bei höher angelegten Spannungen in der Grafik deutlich. Der bestimmte Wert
\\ \\
    \centerline{$E_{\symup{Anr}} = (5.24 \pm 0.11) \symup{eV}$} 
\\ \\
weicht nur um $16.4\%$ vom Literaturwert     
\\ \\
\centerline{$E_{\symup{Lit_Anr}} = 4.5 \symup{eV}$\cite{anr}} 
\\ \\
ab, was auf Grund der vielen Fehlerquellen vertretbar ist.
Anders sieht es bei der Bestimmung der Ionisierungsenergie aus. Der bestimmte Wert von
\\ \\
    \centerline{$E_{\symup{ion}} = 0.57 \symup{eV}$}
\\ \\
weicht um $94.54\%$ vom Literaturwert
\\ \\
    \centerline{$E_{\symup{ion}} = 10.438 \symup{eV}$  \cite{ion}}
\\ \\
ab, was so nicht mit den vorhandenen Fehlerquellen erklärbar ist. Hier muss während der Messung ein grober Fehler unterlaufen sein, da diese 
Diskrepanz sonst nicht erklärbar ist. Wahrscheinlich erscheint dabei ein Fehler in der Skalierung der Geräte, welche sich als schwierig gestaltete,
da es sich bei den verwendeten Geräten um ältere, analoge Modelle handelt. Ein anderer Fehlerursprung besteht aber auch in der Einstellung des 
Spannungsgerätes, welche manuell vorgenommen wird und daher sehr fehleranfällig ist.

