\section{Auswertung}
\label{sec:Auswertung}
    \subsection{Bestimmung der mittleren freien Weglänge}
    Gemäß ??? wird die mittlere freie Weglänge $\overline{w}$ zu den im Experiment verwendeten Temperaturen bestimmt und mit dem Abstand $a = 0.01 \symup{m}$
    zwischen der Kathode und der Beschleunigungselektrode verglichen. Die Weglänge $\overline{w}$ sowie das Verhältnis $\frac{a}{\overline{w}}$
    zu den zu den zugehörigen Temperaturen sind in \autoref{tab:weglaenge} zu sehen. Der Fehler des hier verwendeten Thermometers wird zu 
    der Genauigkeit der Skala auf $0.1 \symup{K}$ abgeschätzt.
    \begin{table}[!htp]
\centering
\caption{Mittlere freie Weglängen zu den verwendeten Temperaturen, sowie dem Verhältnis zum Abstand Kathode-Beschleunigunselektrode }
\label{tab:weglaenge}
\begin{tabular}{c c c c}
\toprule
{{$T$ / K}} & {{Dampfdruck /mbar}} & {{$\overline{w}$ / m}} & {{$\frac{a}{w}$}} \\
\midrule
$298.65$ & $5.51\cdot 10^{-3} $ & $5.26 \cdot 10^{-3}  $ &  $ 1.901$\\
$416.15$ & $3.67 $ & $7.904 \cdot 10^{-6} $  & $  1265 $ \\
$468.65$ & $23.36$ & $1.242 \cdot 10^{-6} $  & $  8054 $ \\
$379.15$ & $0.73 $ & $3.964 \cdot 10^{-5}  $ &  $ 252.3 $ \\
\bottomrule
\end{tabular}
\end{table}
    Die Fehler der Weglänge $\overline{w}$, sowie des Verhältnises $\frac{a}{\overline{w}}$ ergeben sich mittels Fehlerfortpflanzung zu
    \\ \\
    \centerline{$\delta \overline{w} = \delta T \cdot T^2 \cdot \frac{0.000029}{5.5 \cdot 10^7 e^{-\frac{6876}{T}}}$}
    \\ \\ %DIESE FEHLERieHFORMELN SIND NICHT KORREKT ODER VOLLSTÄNDIG

    \subsection{Differentielle Energieverteilung der Elektronen}
    Die differentielle Energieverteilung der Elektronen werden mithilfe der ersten beiden Grafiken aus dem Anhang bestimmt.
    