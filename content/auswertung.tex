\section{Auswertung}
\label{sec:Auswertung}
    \subsection{Bestimmung der mittleren freien Weglänge}
    Gemäß \eqref{eqn:druck-abstand} und \eqref{eqn:saettigung} wird die mittlere freie Weglänge $\overline{w}$ zu den im Experiment verwendeten Temperaturen bestimmt und mit dem Abstand $a = 0.01 \symup{m}$
    zwischen der Kathode und der Beschleunigungselektrode verglichen. Die Weglänge $\overline{w}$ sowie das Verhältnis $\frac{a}{\overline{w}}$
    zu den zu den zugehörigen Temperaturen sind in \autoref{tab:weglaenge} zu sehen. 
    %Der Fehler des hier verwendeten Thermometers wird zu der Genauigkeit der Skala auf $0.1 \symup{K}$ abgeschätzt.
    \begin{table}[!htp]
\centering
\caption{Mittlere freie Weglängen zu den verwendeten Temperaturen, sowie dem Verhältnis zum Abstand Kathode-Beschleunigunselektrode.}
\label{tab:weglaenge}
\begin{tabular}{c c c c}
\toprule
{{$T$ / K}} & {{Dampfdruck /mbar}} & {{$\overline{w}$ / m}} & {{$\frac{a}{w}$}} \\
\midrule
$298.65$ & $5.51\cdot 10^{-3} $ & $5.260 \cdot 10^{-3}  $ &  $ 1.901$\\
$416.15$ & $3.67 $ & $7.904 \cdot 10^{-6} $  & $  1265 $ \\
$468.65$ & $23.36$ & $1.242 \cdot 10^{-6} $  & $  8054 $ \\
$379.15$ & $0.73 $ & $3.964 \cdot 10^{-5}  $ &  $ 252.3 $ \\
\bottomrule
\end{tabular}
\end{table}


    \subsection{Differentielle Energieverteilung der Elektronen}
        \subsubsection{bei Raumtemperatur ($T = 25.5 $ °C)}
        \label{sec:raumtemp}
        Für diesen Auswertungsteil wird die Abbildung 1 aus dem Anhang verwendet. Zuerst muss der Maßstab der x-Achse der Grafik bestimmt werden.
        Dazu wird die maximale gemessene Spannung von $U_{\symup{max}} = 10 \symup{V}$ durch die Länge der gesamten Kurve (24cm) geteilt.
        Dadurch ergibt sich eine Skalierung von $1 \symup{cm} \, \hat{=} \, 0.42 \symup{V}$ . Durch Unterteilung der Achse 
        in kleine Teilstücke, lässt sich für jedes dieser Teilstücke eine Näherung der Steigung bestimmen. Die abgelesenen Steigungen, sowie
        die der zugehörige Spannungswert sind in \autoref{tab:messreihe1} zu sehen und in \autoref{fig:raumtemp} grafisch aufgetragen.
        In der Grafik zeigt die y-Achse die Steigung der integralen Verteilung, welche proportional zur Anzahl Elektronen ist, während die x-Achse
        die Abbremsspannung $U_{\symup{A}}$ zeigt und damit proportional zur Energie der Elektronen ist.
        Wie zu sehen ist haben die meisten Elektronen eine Energie von $9.24$eV. Die angelegte, konstante Beschleunigungsspannung $U_{\symup{B}}$
        ist aber auf $11$V eingestellt. Nach \eqref{eqn:accl-eff} beträgt das Kontaktpotential somit $K_{\symup{1}} = 1.76$V. Die in \autoref{fig:raumtemp} dargestellte 
        Kurve beschreibt zunächst eine relativ konstante Verteilung mit geringen Werten, steigt dann aber steil zum Maximum an, um danach ebenfalls steil
        wieder zu sinken. Aus diesem Peak lässt sich, wie bereits erwähnt, schließen, dass die meisten Elektronen die selbe Energie haben. 
        \begin{table}[!htp]
\centering
\caption{Differentielle Energieverteilung bei Raumtemperatur (25.5 °C).}
\label{tab:messreihe1}
\begin{tabular} {c c}
\toprule
 {{$U_{\symup{A}}$ / V (Abstand bis zum nächsten Eintrag)}} & {{Steigung  / $\frac{\text{Änderung y-Achse}}{\text{Änderung x-Achse}}$}}  \\
\midrule
0.00 & 0.0   \\
0.42 & 0.0   \\
0.84 & 0.1 \\
1.26 & 0.0   \\
1.68 & 0.0   \\
2.10 & 0.1 \\
2.52 & 0.0   \\
2.94 & 0.1 \\
3.36 & 0.1 \\
3.78 & 0.1 \\
4.20 & 0.1 \\
4.62 & 0.1 \\
5.04 & 0.1 \\
5.46 & 0.1 \\
5.88 & 0.2 \\
6.30 & 0.2 \\
6.72 & 0.2 \\
7.14 & 0.3 \\
7.56 & 0.5 \\
7.99 & 0.5 \\
8.40 & 0.7 \\
8.61 & 0.5 \\
8.82 & 0.7 \\
9.03 & 1.3 \\
9.13 & 1.4 \\
9.24 & 2.1 \\
9.35 & 1.9 \\
9.45 & 1.1 \\
9.66 & 0.6 \\
\bottomrule 
\end{tabular}
\end{table}
        \begin{figure}
          \centering
          \includegraphics{raumtemp.pdf}
          \caption{Differentielle Energieverteilung bei Raumtemperatur($T = 25.5$°C)}
          \label{fig:raumtemp}
        \end{figure}


        \subsubsection{bei $T = 143$ °C}
        Bei der Energieverteilungsmessung bei 143°C wird genauso verfahren wie schon im vorherigen Kapitel (siehe \autoref{sec:raumtemp}).
        Die aus Abbildung 2 aus dem Anhang abgelesenen Spannungen $U_{\symup{A}}$ und die zugehörigen Steigungen sind in \autoref{tab:messreihe2} aufgelistet
        und in \autoref{fig:messreihe2} grafisch dargestellt. Zu sehen ist zuerst eine konstante Verteilung, die jedoch steil absinkt und von da an nur noch eine
        geringe (fast) konstante Verteilung zeigt. Hier zeigt sich, dass die meisten Elektronen eine geringere Energie haben, allerdings auch stärker
        im Energiespektrum verteilt sind. Der starke Abfall der Anzahl an Elektronen bei ungefähr $U_{\symup{A}} = 4$V lässt sich mithlife von 
        \autoref{sec:franck} erklären, da bei dieser ungefähr dieser Spannung die Quecksilberatome bereits angeregt werden können.
        \begin{table}[!htp]
\centering
\caption{Differentielle Energieverteilung bei 143°C.}
\label{tab:messreihe2}
\begin{tabular}{c c}
\toprule
{$U_{\symup{A}}$/ V (Abstand bis zum nächsten Eintrag)} & {Steigung  / $\frac{\text{Änderung y-Achse}}{\text{Änderung x-Achse}}$} \\
\midrule
0.00 & 0.9 \\
0.42 & 1.0 \\
0.84 & 1.0 \\
1.26 & 1.0 \\
1.68 & 1.0 \\
2.10 & 1.1 \\
2.52 & 1.0 \\
2.94 & 1.0 \\
3.36 & 1.0 \\
3.78 & 0.6 \\
4.20 & 0.4 \\
4.62 & 0.1 \\
5.46 & 0.1 \\
6.30 & 0.1 \\
6.72 & 0.1 \\
7.14 & 0.1 \\
7.56 & 0.1 \\
8.40 & 0.1 \\
8.82 & 0.1 \\
9.24 & 0.1 \\
9.66 & 0.0 \\
\bottomrule
\end{tabular}
\end{table}
         \begin{figure}
          \centering
          \includegraphics{messreihe2.pdf}
          \caption{Differentielle Energieverteilung bei $T = 143$°C}
          \label{fig:messreihe2}
        \end{figure}
    \subsection{Franck-Hertz-Kurve} 
    \label{sec:franck}
    Die aufgenomme Franck-Hertz-Kurve ist in Abbildung 3 des Anhangs zu sehen. 
    Weitere Franck-Hertz-Kurven sind dem Anhang beigefügt, werden aber in der Auswertung nicht weiter beachtet,
    da die ausgewählte Kurve die deutlichsten Extrema aufweist. Zur Bestimmung der ersten Anregungsenergie des Quecksilbers
    werden die Abstände der Extrema der Franck-Hertz-Kurve aufgenommen und gemittelt. In dieser Abbildung enstspricht 1cm hier 2.45V. Die Spannungsdifferenzen
    sind in \autoref{tab:franck} aufgestellt. Die Mittelung wird durch
    \begin{equation}
      U_{\symup{Anr}} = \frac{1}{8} \sum_{i = 1}^{8} (U_{\symup{i+1}} - U_{\symup{i}})
    \end{equation}  
    und der zugehörige Fehler durch 
    \begin{equation}
      \Delta U_{\symup{Anr}} = \frac{1}{\sqrt{56}} \sqrt{ \sum_{i = 1}^{8} (U_{\symup{i}} - U_{\symup{Anr}}) }
    \end{equation}
    errechnet. Hierdurch ergibt sich eine erste Anregungsspannung von 
    \\ \\
    \centerline{$U_{\symup{Anr}} = (5.24 \pm 0.11) \symup{V}$}
    \\ \\
    beziehungsweise eine erste Anregungsenergie von
    \\ \\
    \centerline{$E_{\symup{Anr}} = (5.24 \pm 0.11) \symup{eV}$.}
    \\ \\  
    Durch Gleichung \eqref{eqn:photon-energie} und dem Zusammenhang $\lambda = \frac{c}{\nu}$ ergibt sich die Wellenlänge des emmitierten Lichts zu
    \\ \\
    \centerline{$\lambda = (237 \pm 5) nm$},
    \\ \\
    wobei sich der Fehler durch Fehlerfortpflanzung zu
    \begin{equation}
      \Delta \lambda = \frac{c \cdot h}{E_{\symup{Anr}}} \cdot \Delta E_{\symup{Anr}}
    \end{equation}
    berechnet. Das emmitierte Licht liegt damit im ultravioletten Bereich. 
    Das erste Spannungsmaximum liegt bei 
    \\ \\
    \centerline{$U_{\symup{1}} = 5.70$V.}
    \\ \\
    Da $U_{\symup{1}} = K + U_{\symup{Anr}} $ gilt, lässt sich das Kontaktpotential zu 
    \\ \\
    \centerline{$K_{\symup{2}} = (0.46 \pm 0.11)$ V}
    \\ \\
    bestimmen.
    

    \begin{table}[!htp]
\centering
\caption{Abstände der Maxima der Franck-Hertz-Kurve.}
\label{tab:franck}
\begin{tabular}{c c c}
\toprule
{i} & {Abstand / cm} & { Abstand $U_{\symup{i+1}} - U_{\symup{i}}$ / V} \\
\midrule
1 & $2.0$ & $4.91$ \\
2 & $2.0$ & $4.91$ \\
3 & $2.1$ & $5.14$ \\
4 & $2.2$ & $5.39$ \\
5 & $2.1$ & $5.14$ \\
6 & $2.1$ & $5.14$ \\
7 & $2.2$ & $5.39$ \\
8 & $2.4$ & $5.89$ \\
\bottomrule
\end{tabular}
\end{table}


    \subsection{Ionisierungsenergie} 
    Die Ionisierungsenergie von Quecksilber wird mithilfe von Abbildung 4 aus dem Anhang bestimmt. 
    Hierzu wird eine Asymptote in die Grafik eingezeichnet. In dieser Grafik enstspricht 1cm $1$V.
    Die Schnittstelle der Asymptote mit der x-Achse liegt somit bei $U_{\symup{Schnitt}} = 10$V.
    Zusammen mit den zuvor bestimmten Kontaktpotentialen
    ergeben sich die Ionisierungsenergien somit zu
    \\ \\
    \centerline{$E_{\symup{ion, 1}} = 8.24 \symup{eV}$,}
    \\ \\
    \centerline{$E_{\symup{ion, 2}} = (9.54 \pm 0.11) \symup{eV}$.}