\section{Auswertung}
\label{sec:Auswertung}
    \subsection{Bestimmung der mittleren freien Weglänge}
    Gemäß ??? wird die mittlere freie Weglänge $\overline{w}$ zu den im Experiment verwendeten Temperaturen bestimmt und mit dem Abstand $a = 0.01 \symup{m}$
    zwischen der Kathode und der Beschleunigungselektrode verglichen. Die Weglänge $\overline{w}$ sowie das Verhältnis $\frac{a}{\overline{w}}$
    zu den zu den zugehörigen Temperaturen sind in \autoref{tab:weglaenge} zu sehen. Der Fehler des hier verwendeten Thermometers wird zu 
    der Genauigkeit der Skala auf $0.1 \symup{K}$ abgeschätzt.
    \begin{table}[!htp]
\centering
\caption{Mittlere freie Weglängen zu den verwendeten Temperaturen, sowie dem Verhältnis zum Abstand Kathode-Beschleunigunselektrode.}
\label{tab:weglaenge}
\begin{tabular}{c c c c}
\toprule
{{$T$ / K}} & {{Dampfdruck /mbar}} & {{$\overline{w}$ / m}} & {{$\frac{a}{w}$}} \\
\midrule
$298.65$ & $5.51\cdot 10^{-3} $ & $5.260 \cdot 10^{-3}  $ &  $ 1.901$\\
$416.15$ & $3.67 $ & $7.904 \cdot 10^{-6} $  & $  1265 $ \\
$468.65$ & $23.36$ & $1.242 \cdot 10^{-6} $  & $  8054 $ \\
$379.15$ & $0.73 $ & $3.964 \cdot 10^{-5}  $ &  $ 252.3 $ \\
\bottomrule
\end{tabular}
\end{table}
   % Die Fehler der Weglänge $\overline{w}$, sowie des Verhältnises $\frac{a}{\overline{w}}$ ergeben sich mittels Fehlerfortpflanzung zu
    %\\ \\
   % \centerline{$\delta \overline{w} = \delta T \cdot T^2 \cdot \frac{0.000029}{5.5 \cdot 10^7 e^{-\frac{6876}{T}}}$}
    %\\ \\ %DIESE FEHLERieHFORMELN SIND NICHT KORREKT ODER VOLLSTÄNDIG

    \subsection{Differentielle Energieverteilung der Elektronen}
        \subsubsection{bei Raumtemperatur ($T = 25.5 $ °C)}
        Für diesen Auswertungsteil wird die Abbildung 1 aus dem Anhang verwendet. Zuerst muss der Maßstab der x-Achse der Grafik bestimmt werden.
        Dazu wird die maximale gemessene Spannung von $U_{\symup{max}} = 10 \symup{V}$ durch die Länge der gesamten Kurve (24cm) geteilt.
        Dadurch ergibt sich eine Skalierung von $1 \symup{cm} \hat{=} 0.42 \symup{V}$ . Durch Unterteilung der Achse 
        in kleine Teilstücke, lässt sich für jedes dieser Teilstücke eine Näherung der Steigung bestimmen. Die abgelesenen Steigungen, sowie
        die der zugehörige Spannungswert sind in \autoref{tab:messreihe1} zu sehen.
        \begin{table}[!htp]
\centering
\caption{Differentielle Energieverteilung bei Raumtemperatur (25.5 °C).}
\label{tab:messreihe1}
\begin{tabular} {c c}
\toprule
 {{$U_{\symup{A}}$ / V (Abstand bis zum nächsten Eintrag)}} & {{Steigung  / $\frac{\text{Änderung y-Achse}}{\text{Änderung x-Achse}}$}}  \\
\midrule
0.00 & 0.0   \\
0.42 & 0.0   \\
0.84 & 0.1 \\
1.26 & 0.0   \\
1.68 & 0.0   \\
2.10 & 0.1 \\
2.52 & 0.0   \\
2.94 & 0.1 \\
3.36 & 0.1 \\
3.78 & 0.1 \\
4.20 & 0.1 \\
4.62 & 0.1 \\
5.04 & 0.1 \\
5.46 & 0.1 \\
5.88 & 0.2 \\
6.30 & 0.2 \\
6.72 & 0.2 \\
7.14 & 0.3 \\
7.56 & 0.5 \\
7.99 & 0.5 \\
8.40 & 0.7 \\
8.61 & 0.5 \\
8.82 & 0.7 \\
9.03 & 1.3 \\
9.13 & 1.4 \\
9.24 & 2.1 \\
9.35 & 1.9 \\
9.45 & 1.1 \\
9.66 & 0.6 \\
\bottomrule 
\end{tabular}
\end{table}