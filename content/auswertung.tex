\section{Auswertung}
\label{sec:Auswertung}
    \subsection{Bestimmung der mittleren freien Weglänge}
    Gemäß ??? wird die mittlere freie Weglänge $\overline{w}$ zu den im Experiment verwendeten Temperaturen bestimmt und mit dem Abstand $a = 0.01 \symup{m}$
    zwischen der Kathode und der Beschleunigungselektrode verglichen. Die Weglänge $\overline{w}$ sowie das Verhältnis $\frac{a}{\overline{w}}$
    zu den zu den zugehörigen Temperaturen sind in \autoref{tab:weglaenge} zu sehen. 
    %Der Fehler des hier verwendeten Thermometers wird zu der Genauigkeit der Skala auf $0.1 \symup{K}$ abgeschätzt.
    \begin{table}[!htp]
\centering
\caption{Mittlere freie Weglängen zu den verwendeten Temperaturen, sowie dem Verhältnis zum Abstand Kathode-Beschleunigunselektrode }
\label{tab:weglaenge}
\begin{tabular}{c c c c}
\toprule
{{$T$ / K}} & {{Dampfdruck /mbar}} & {{$\overline{w}$ / m}} & {{$\frac{a}{w}$}} \\
\midrule
$298.65$ & $5.51\cdot 10^{-3} $ & $5.26 \cdot 10^{-3}  $ &  $ 1.901$\\
$416.15$ & $3.67 $ & $7.904 \cdot 10^{-6} $  & $  1265 $ \\
$468.65$ & $23.36$ & $1.242 \cdot 10^{-6} $  & $  8054 $ \\
$379.15$ & $0.73 $ & $3.964 \cdot 10^{-5}  $ &  $ 252.3 $ \\
\bottomrule
\end{tabular}
\end{table}


    \subsection{Differentielle Energieverteilung der Elektronen}
        \subsubsection{bei Raumtemperatur ($T = 25.5 $ °C)}
        \label{sec:raumtemp}
        Für diesen Auswertungsteil wird die Abbildung 1 aus dem Anhang verwendet. Zuerst muss der Maßstab der x-Achse der Grafik bestimmt werden.
        Dazu wird die maximale gemessene Spannung von $U_{\symup{max}} = 10 \symup{V}$ durch die Länge der gesamten Kurve (24cm) geteilt.
        Dadurch ergibt sich eine Skalierung von $1 \symup{cm} \hat{=} 0.42 \symup{V}$ . Durch Unterteilung der Achse 
        in kleine Teilstücke, lässt sich für jedes dieser Teilstücke eine Näherung der Steigung bestimmen. Die abgelesenen Steigungen, sowie
        die der zugehörige Spannungswert sind in \autoref{tab:messreihe1} zu sehen und in \autoref{fig:messreihe1} grafisch aufgetragen.
        In der Grafik zeigt die y-Achse die Steigung der integralen Verteilung, welche proportional zur Anzahl Elektronen ist, während die x-Achse
        die Abbremsspannung $U_{\symup{A}}$ zeigt und damit proportional zur Energie der Elektronen ist.
        Wie zu sehen ist haben die meisten Elektronen eine Energie von $9.24$eV. Die angelegte, konstante Beschleunigungsspannung $U_{\symup{B}}$
        ist aber auf $11$V eingestellt. Nach ??? beträgt das Kontaktpotential somit $K = 1.76$V. Die in \autoref{fig:messreihe1} dargestellte 
        Kurve beschreibt zunächst eine relativ konstante Verteilung mit geringen Werten, steigt dann aber steil zum Maximum an, um danach ebenfalls steil
        wieder zu sinken.
        \begin{table}[!htp]
\centering
\caption{Differentielle Energieverteilung bei Raumtemperatur (25.5 °C)}
\label{tab:messreihe1}
\begin{tabular} {c c c}
\toprule
{{$\delta U_{\symup{A}}$ / V}} & {{$U_{\symup{A}}$ / V}} & {{Steigung  / $\frac{\text{Änderung y-Achse}}{\text{Änderung x-Achse}}$}}  \\
\midrule
1    & 0    & 0   \\
1    & 0.42 & 0   \\
1    & 0.84 & 0.1 \\
1    & 1.26 & 0   \\
1    & 1.68 & 0   \\
1    & 2.10 & 0.1 \\
1    & 2.52 & 0   \\
1    & 2.94 & 0.1 \\
1    & 3.36 & 0.1 \\
1    & 3.78 & 0.1 \\
1    & 4.20 & 0.1 \\
1    & 4.62 & 0.1 \\
1    & 5.04 & 0.1 \\
1    & 5.46 & 0.1 \\
1    & 5.88 & 0.2 \\
1    & 6.30 & 0.2 \\
1    & 6.72 & 0.2 \\
1    & 7.14 & 0.3 \\
1    & 7.56 & 0.5 \\
1    & 7.99 & 0.5 \\
1    & 8.40 & 0.7 \\
0.5  & 8.61 & 0.5 \\
0.5  & 8.82 & 0.7 \\
0.5  & 9.03 & 1.3 \\
0.25 & 9.13 & 1.4 \\
0.25 & 9.24 & 2.1 \\
0.25 & 9.35 & 1.9 \\
0.25 & 9.45 & 1.1 \\
0.50 & 9.66 & 0.6 \\
\bottomrule 
\end{tabular}
\end{table}
        \begin{figure}
          \centering
          \includegraphics{raumtemp.pdf}
          \caption{Differentielle Energieverteilung bei Raumtemperatur($T = 25.5$°C)}
          \label{fig:raumtemp}
        \end{figure}


        \subsubsection{bei $T = 143$ °C}
        Bei der Energieverteilungsmessung bei 143°C wird genauso verfahren wie schon im vorherigen Kapitel (siehe \autoref{sec:raumtemp}).
        Die aus Abbildung 2 aus dem Anhang abgelesenen Spannungen $U_{\symup{A}}$ und die zugehörigen Steigungen sind in \autoref{tab:messreihe2} aufgelistet
        und in \autoref{fig:messreihe2} grafisch dargestellt. Zu sehen ist zuerst eine konstante Verteilung, die jedoch steil absinkt und von da an nur noch eine
        geringe (fast) konstante Verteilung zeigt. (-GENAUER!!!)
        \input{../content/messreihe2.tex}
         \begin{figure}
          \centering
          \includegraphics{messreihe2.pdf}
          \caption{Differentielle Energieverteilung bei $T = 143$°C}
          \label{fig:messreihe2}
        \end{figure}
    \subsection{Franck-Hertz-Kurve} 
    Die aufgenomme Franck-Hertz-Kurve ist in Abbildung 3 zu sehen. Weitere Franck-Hertz-Kurven sind dem Anhang beigefügt, werden aber in der Auswertung nicht 
    weiter beachtet, da die ausgewählte Kurve die deutlichsten Extrema aufweist.   